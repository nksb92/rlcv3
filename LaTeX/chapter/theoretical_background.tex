\section{Theoretical Background}
\label{sec:theoretical_background}
\subsection{Embedded Systems}
An embedded system is fundamentally a computer system designed for a specific, dedicated function within a larger mechanical or electrical system. It comprises a combination of hardware, typically centered around a processor or \ac{MCU}, memory for storing software and data, and peripheral \ac{IO} devices for interacting with the external environment or the larger system it controls. Unlike general-purpose computers, embedded systems are often constrained by factors such as cost, size, power consumption, and real-time performance requirements, and their software, known as firmware, is specifically tailored to their dedicated task. The \ac{LED} controller developed in this work is an example of such an embedded system.

\subsection{RGB LEDs and Control Methods}
\ac{RGB} \ac{LED}s are semiconductor light sources that combine three individual \ac{LED} elements: red, green, and blue within a single package. By varying the intensity of each primary color, a wide spectrum of colors, including white, can be produced through additive color mixing. Controlling these \ac{LED}s typically involves one of two primary methods relevant to this project. The first method uses \ac{PWM} signals applied independently to each color channel of non-addressable \ac{LED}s, allowing for intensity control of the entire strip or fixture as a single unit. The second method employs addressable \ac{LED}s, where each \ac{LED} (or a small group) incorporates an \ac{IC}. These \ac{IC}s allow the \ac{LED}s to be connected in series and controlled individually via a serial data protocol, enabling complex effects and patterns across the length of an \ac{LED} strip or within a fixture. The choice of control method depends on the specific application requirements, balancing factors like cost, complexity, and the desired level of control granularity.

\subsection{Communication Protocols}
The embedded \ac{LED} controller supports industry-standard communication protocols for receiving lighting control data, enabling integration into various professional and hobbyist setups. Wired communication is primarily handled via the \ac{DMX} protocol, technically known as DMX512. \ac{DMX} is a unidirectional, serial digital protocol transmitted over an \ac{RS485} physical layer, designed for controlling stage lighting and effects. It allows for the addressing and control of up to 512 individual channels within a single "universe," typically corresponding to parameters like intensity or color values for different fixtures. The controller implements \ac{DMX} reception capabilities, allowing it to act as a DMX-controllable device.\\

For network-based control, particularly over wireless networks, the controller utilizes the Art-Net protocol. Art-Net is designed to transmit \ac{DMX} data packets over \ac{IP} networks, leveraging standard networking infrastructure like Ethernet or \ac{WLAN}. This allows for significantly larger systems spanning multiple \ac{DMX} universes and facilitates control from software or consoles connected to the same network. The controller's integrated \ac{WLAN} interface enables it to receive Art-Net commands wirelessly, translating them into the appropriate \ac{LED} output signals. The support for both \ac{DMX} and Art-Net provides flexibility in how the controller is integrated and controlled within a lighting system.

\subsection{User Interface Components}
The primary components facilitating user interaction with the embedded \ac{LED} controller constitute the device's \ac{HMI}. These components allow the user to configure settings, select operating modes, and view status information directly on the device.\\

The main input mechanism is a rotary encoder combined with an integrated push button. This single component serves multiple roles: rotation is used for navigating through menu options or adjusting parameter values, a short press typically confirms a selection or cycles through options within a submenu, and a long press is used to switch between the main menu and submenu levels. This provides a compact and intuitive method for controlling the device's functions without requiring numerous buttons. \\

Visual feedback is provided by a monochrome \ac{OLED} display. This display renders the menu structure, shows current parameter values (such as \ac{HSV} or \ac{RGB} levels, \ac{DMX} addresses, or Art-Net settings), and indicates system status, potentially including connectivity icons or confirmation messages like "Saved". The combination of the rotary encoder input and the \ac{OLED} display output forms the complete local user interface for the controller.

\subsection{Power Supply}
The embedded controller is designed to operate from a flexible main power source, accepting a \ac{DC} input voltage ranging from 12 to 24 Volts via a dedicated connector. Internally, an integrated \ac{DCDC} voltage converter steps this input down to a regulated 5 Volts. This stable 5 Volt supply rail powers the core processing components, including the \ac{MCU} and the \ac{OLED} display. Additionally, the \ac{USB} Type C interface, primarily intended for programming and debugging, can also supply power to the board, typically at 5 Volts, although the main operational power is expected from the 12-24 Volt input.

\subsection{Additive Manufacturing (3D Printing)}\label{subsec:theoretical_3d_printing}
Additive manufacturing, commonly known as \ac{3D} printing, encompasses a set of processes used to create \ac{3D} objects by adding material layer by layer. This approach contrasts fundamentally with traditional subtractive manufacturing methods, which involve removing material from a larger block (e.g., milling or turning). \ac{3D} printing typically starts with a digital model, often created using \ac{CAD} software, which is then sliced into thin cross-sections. The printer deposits, fuses, or solidifies material (such as plastics, resins, metals, or composites) according to these cross-sections, gradually building up the final object. This technology is particularly advantageous for rapid prototyping, creating complex geometries, producing customized parts, and enabling low-volume production runs, such as potentially fabricating custom enclosures or mounting hardware for embedded systems like the one described in this work.

\subsection{Aluminum Extrusion Profiles}\label{subsec:theoretical_aluminum_profiles}
Aluminum extrusion profiles are commonly utilized in mechanical design and construction due to their advantageous properties, including a high strength-to-weight ratio, corrosion resistance, and excellent thermal conductivity. The extrusion process allows for the creation of complex cross-sectional shapes with integrated features like channels, slots, and mounting points. These features make aluminum profiles highly suitable for creating modular and customizable enclosures and structures. In the context of lighting applications, particularly for linear fixtures like the tube light variants developed in this project, these profiles serve multiple purposes. They provide a robust yet lightweight housing, act as an effective heat sink for dissipating thermal energy generated by the \ac{LED}s and electronics, and often include channels designed to hold the \ac{LED} strips, diffusers, and associated \ac{PCB}s securely in place, contributing to a clean and integrated final design.