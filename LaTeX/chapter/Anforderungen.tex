\section{Aufgabenstellung}\label{kap:aufgabenstellung}
\subsection{Aktueller Stand}\label{kap:atueller_stand}
In der Linker-Konfigurationsdatei werden Speicherregionen für das \ac{ROM} und das \ac{RAM} angelegt.
Diesen Regionen werden zur Separierung unterschiedliche Softwareblöcke zugewiesen.
Für die Zuweisung der Softwareartefakte, des \acs{ROM}-Programmcodes und Konstanten sowie \acs{RAM}-Variablen gibt es verschiedene Möglichkeiten.
Der \acs{ROM}-Programmcode kann über die Compileroptionen der \ac{IDE} einem Modul zugeordnet werden.
\acs{ROM}-Konstanten und \acs{RAM}-Variablen können über eine spezielle Direktive direkt im Sourcecode verlinkt werden.
Die Syntax der Linker-Konfigurationsdatei bietet eine weitere Direktive zur Zuordnung an.
Dadurch können \acs{ROM}-Programmcode, \acs{ROM}-Konstanten und \acs{RAM}-Variablen von Drittanbietern in den Speicherbereichen platziert werden.
Drittanbieter sind in diesem Zusammenhang auch Softwaremodule von Viessmann, welche als Bibliothek eingesetzt werden. \\

Grundstein der Thesis ist das Analysetool \verb*|Linker| \verb*|File| \verb*|Analyser|.
Für den Nutzer der Software bietet es eine graphische Benutzeroberfläche.
Darüber kann eine Linker-Konfigurationsdatei eingelesen und die darin definierten Speicherregionen angezeigt werden.
In Abbildung \ref{fig:anwendungsübersicht} können die drei separaten Fenster der Anwendung gefunden werden.\\

\begin{figure}[H]
	\centering
	\subfloat[\label{fig:hauptanwendungsfenster}Hauptanwendungsfenster]{{\includegraphics[width=.75\linewidth]{graphics/gui_main_window_regions} }} \\
	\subfloat[\label{fig:mpu_konfiguration}MPU-Konfiguration]{{\includegraphics[width=0.53\linewidth]{graphics/gui_options_mpu} }} \\
	\subfloat[\label{fig:mpu_visualisierung}MPU-Visualisierung]{{\includegraphics[width=0.75\linewidth]{graphics/gui_show_window} }} 
	\qquad	
	\caption{verschiedene Fenster der graphischen Benutzeroberfläche}
	\label{fig:anwendungsübersicht}
\end{figure}

Eine wichtige Funktion der Software ist das Erstellen von Konfigurationen für die \ac{MPU} eines Mikrocontrollers.
Dafür kann eine oder mehrere Speicherregionen im Hauptanwendungsfenster (Abbildung \ref{fig:hauptanwendungsfenster}) ausgewählt werden.
Weiter ist es wichtig, dass für das Erstellen von Konfigurationen für die \acs{MPU} gewisse Regeln eingehalten werden. 
Diese werden im Hintergrund geprüft.
Sollten dabei Verstöße auftreten, wird dies dem Nutzer als Fehlermeldung zurückgeben.
Über die Optionen (Abbildung \ref{fig:mpu_konfiguration}) kann der Speicherort der Datei oder auch die verwendete Architektur des Mikrocontrollers ausgewählt werden.
Im Hintergrund wird mit einem effizientem Algorithmus die Konfiguration erstellt.
Dabei werden die Bits der entsprechenden Register des Controllers gesetzt. 
Auf dem Mikrocontroller stehen dafür verschiedene Regionen zur Verfügung.
Gespeichert wird diese Konfiguration in einer Header-Datei.\\

In einem weiteren Fenster können die Konfigurationen visualisiert werden.
Abbildung \ref{fig:mpu_visualisierung} zeigt die Darstellung der Regionen und Subregionen einer solchen Konfiguration.
Dieses Fenster ermöglicht einen einfachen Überblick der Konfiguration.
Verschiedene Farben zeigen dabei die aktiven bzw. inaktiven Subregionen an. \\

\subsection{Projektanforderungen}\label{kap:projektanforderungen}
In der heutigen Zeit spielen \ac{GUI}s eine entscheidende Rolle als Schnittstelle zwischen Nutzer und Software.
So auch beim Softwaretool \verb*|Linker| \verb*|File| \verb*|Analyser|.
Dieses Softwaretool soll erweitert werden.
Bei der bestehenden Software sollen die Stärken und Schwächen analysiert werden.
Abgeleitet davon werden Erweiterungen geplant und Anforderungen gesetzt.
Die bestehende Funktionalität wird bereits im vorherigen Kapitel (\ref{kap:atueller_stand} \nameref{kap:atueller_stand}) erläutert.
Die folgenden drei Punkte stellen die resultierenden Anforderungen an die Erweiterungen dar. \\

\begin{description}[style=nextline]
	\item[Gruppierung der Einzelfenster]
	Die verschiedenen Fenster der Software sollen für eine verbesserte Übersicht in einer Hauptanwendung vereint werden.
	Die Funktion dieser bereits vorhandenen Fenster soll bestehen bleiben.
	In diesem Hauptanwendungsfenster sollen alle Funktionen gebündelt bedienbar sein.
	Einzelne Fenster sollen weiterhin für das Anzeigen von Informations-, Warn- oder Fehlermeldungen genutzt werden können.
	
	\item[Manipulation der Speicherbereiche]
	Die Definition der Speicherbereiche in der Linker-Konfigurationsdatei ist unübersichtlich.
	Die Änderungen der Bereichsadressen oder Bereichsgröße sind aufwändig und fehleranfällig.
	Daher soll es über das Werkzeug möglich sein, dies zu ändern.
	Die Linker-Konfigurationsdatei soll mit den neuen Informationen aktualisiert und die Änderungen übernommen werden.
	Über das Ändern der bereits vorhandenen Speicherregionen hinaus soll es weitere Funktionen für das Erstellen und Löschen von Regionen geben.
	Auch diese Veränderung der Speicherregionen soll in die Linker-Konfigurationsdatei eingebracht werden.
	Die Segmentierung der Speicherregionen soll dabei berücksichtigt werden.
	Neue Segmente sollen über die \ac{GUI} angelegt werden können.
	Diese Erweiterung geht mit einer Änderung bzw. Anpassung der Elemente der \ac{GUI} einher.
	Dafür müssen Widgets angepasst, hinzugefügt oder gelöscht werden.
	
	\item[Artefaktzuweisung der Speicherregionen]
	Die Software-Artefakte sollen in den Speicherregionen platziert werden.
	Diese Zuweisung erfolgt aktuell manuell durch einen Entwickler.
	Dieser Schritt soll automatisiert werden.
	Abgeleitet aus der Norm DIN EN 61508:2011-02 müssen alle Software-Artefakte einem Speicherbereich zugeordnet sein.
	Das Automatisieren dieses Schrittes soll Fehler vermeiden und sicherstellen, dass jedes Artefakt zugeordnet ist.
	Ein gegebenes Software-Projekt soll dabei automatisch durchsucht werden.
	Dabei soll ein nicht zugewiesenes Artefakt gekennzeichnet werden.
\end{description}

Für die beschriebenen Anforderungen ist die Einarbeitung in die Syntax der Linker-Konfigurationsdateien essentiell.
Weiter wird ein klares Verständnis für den Ablauf der Toolchain und besonders die Aufgabe des Linkers benötigt.
Eine Einarbeitung und Recherche ist dafür wichtig.\\

Im folgenden Kapitel wird zunächst auf den Aufbau der Software-Projekte einer \ac{ECU} eingegangen.
Darin wird die Verzeichnisstruktur erläutert.
Die Toolchain für das Überführen von Programmcode zu einer ausführbaren Datei folgt danach.
Auf die besondere Rolle des Linkers und dessen Konfiguration wird ab Kapitel \textit{\ref{kap:linker} \nameref{kap:linker}} (Seite \pageref{kap:linker}) eingegangen.
Mit den Grundlagen dieser Kapitel werden anschließend die Erweiterungen beschrieben.
Zunächst wird allerdings auf die Werkzeuge für die Entwicklung sowie die Best Practices eingegangen.
Kapitel \ref{kap:zusammenfassung} fasst die Thesis zusammen.
Darin wird auch die komplette Funktionalität der Software-Anwendung beschrieben.
Es folgt eine Aufführung der Projektergebnisse und zum Abschluss eine Zukunftsaussicht.
