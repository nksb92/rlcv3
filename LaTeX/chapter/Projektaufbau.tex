\section{Software einer ECU}\label{kap:projektaufbau}
\subsection{Verzeichnisstruktur}
Ein gut strukturiertes Verzeichnis ist entscheidend für die Effizienz und Wartbarkeit eines Softwareprojektes.
Es soll dem Entwickler dabei helfen, einen Überblick zu behalten und teamübergreifend zusammenzuarbeiten.
Die eingebetteten Steuergeräte von Viessmann werden hauptsächlich mit der Programmiersprache C und C++ programmiert. 
Vereinzelt wird Assembler verwendet.
Abbildung \ref{fig:verzeichnisstruktur} zeigt dabei den groben Aufbau eines solchen Verzeichnisses.
Diese Struktur ist aus dem Projektverzeichnis der \ac{BMCU} abgeleitet.
Auf der rechten Seite der Abbildung sind die Ordner, auf der Linken die Dateien dargestellt. \\

\begin{figure}[H]
	\centering
	\includegraphics[width=0.5\linewidth]{graphics/verzeichnisstruktur}
	\caption{Verzeichnisstruktur eines C-Projektes}
	\label{fig:verzeichnisstruktur}
\end{figure}

Ein Software-Projekt wird in Module unterteilt.
Dies ist notwendig, um logische Funktionalitäten zu gruppieren.
Für die Zergliederung von Projekten gibt es unterschiedliche Vorgehensweisen.
Bei der Bottom-Up-Methode werden Einzelaufgaben zu einer Gesamtaufgabe zusammengefasst.
Als Zweites gibt es die Top-Down-Methode.
Dabei werden die Gesamtaufgaben von oben nach unten in kleinere Teilaufgaben zerlegt.
Eine Mischung aus beiden Methoden ist natürlich auch möglich \cite{Garcia2017-op}. \\

Logisch zusammenhängende Module bestehen aus Source- und Header-Dateien.
Dabei beschreiben typischerweise eine Header- und eine Source-Datei ein in sich abgeschlossenes Modul mit entsprechender Funktionalität.
Dies erhöht die Übersichtlichkeit des Programms und ist ab einer gewissen Projektgröße unabdingbar.
Außerdem wird die Zeit für das Kompilieren eines Projektes reduziert, weil nur noch die geänderten Dateien neu übersetzt werden müssen \cite{Klima2010-br}.
Weiter können in sich abgeschlossene Module über verschiedene Projekte hinweg genutzt werden.\\

Eine Header-Datei ist dabei wie eine Bauanleitung für den Compiler.
Sie werden in dem Ordner \verb*|include/| (Abbildung \ref{fig:verzeichnisstruktur}) abgelegt.
Darin werden ausgewählte Funktionen für die Steuerungen beschrieben \cite{Klima2010-br}.
Diese fungieren als Schnittstellen zu anderen Modulen.
Weiter werden dort Deklarationen für die Programmierung hinterlegt.
Besonders bei der Programmierung mit Mikrocontrollern ist der Zugriff auf Steuer- und Statusregister wichtig.
Diese Festen Hardwareadressen werden typischerweise in Header-Dateien über Definitionen angelegt \cite{Wust2010-jg}.
Oftmals werden direkt Bitmasken für einen einfacheren Zugriff angefertigt.
Header-Dateien werden für das Kompilieren von Source-Dateien benötigt, da die Schnittstellen nach außen darin beschrieben sind.\\

Die Source-Dateien enthalten den eigentlichen Quellcode.
Diese Dateien werden in dem Ordner \verb*|src/| (Abbildung \ref{fig:verzeichnisstruktur}) abgelegt.
Darin ist die Funktionalität implementiert.
Variablen, die speziell in diesem Modul benötigt werden, können auch in der Source-Datei angelegt werden \cite{Garcia2017-op}.
Die zugehörige Header-Datei muss in die Source-Datei über eine Präprozessor-Anweisung eingebunden werden. \\

Dieser Quelltext muss vom Compiler in ausführbaren Programmcode umgewandelt werden.
Dabei entstehen Objektdateien.
Wie der genaue Ablauf dabei ist, kann im nächsten Kapitel \textit{\ref{fig:toolchain} \nameref{fig:toolchain}} gefunden werden.
Die Objektdateien werden in dem Ordner \verb*|out/| abgelegt.
Wichtig sind diese Dateien besonders während des Linking-Prozesses. \\

Die weiteren Ordner bilden die Struktur für die Dokumentation, externe und interne Bibliotheken sowie für Tests.
Für ein Projekt sind außerdem spezielle Dateien im Verzeichnis nötig.
Die Dateien sind in der Abbildung \ref{fig:verzeichnisstruktur} auf der linken Seite dargestellt.
Zunächst gibt es das Makefile.
Diese Datei wird typischerweise jedoch nicht zwangsweise in C-Projekten verwendet.
Darin werden Pfadangaben und andere Steuerparameter definiert.
Diese Datei steuert die Arbeit des Compilers \cite{Schmitt2019-nd}.
Unterschiedliche Compiler haben dafür unterschiedliche Dateitypen und Syntaxen.\\

In der \verb*|README.md| werden wichtige Informationen zum Verzeichnis, Projekt und jeweiligen Software festgehalten.
Sie dient als Einführung und Anleitung für das Projekt.
Ein Nutzer sollte diese Datei lesen, wenn er das erste Mal auf das Projekt zugreift.
Darin können nämlich Fragen zur Installation oder Verwendung der Software direkt geklärt werden \cite{inet:ionos}.
Weiter verbessert eine gut geschriebene \verb*|README.md|-Datei die Zugänglichkeit und Benutzerfreundlichkeit des Projekts. \\

Das Verzeichnis wird nicht nur lokal gespeichert, sondern unter Versionskontrolle auf einem Server abgelegt.
Verwendet wird dabei die Versionsverwaltung via Git \cite{inet:git}.
Das Verzeichnis oder \verb*|Repository| kann von unterschiedlichen Entwicklern geklont und bearbeitet werden \cite{Preisel2017-eu}.
Es gibt Dateien oder Ordner, die nicht versioniert gespeichert werden müssen.
Mit der Datei \verb*|.gitignore| können Unterverzeichnisse, Einzeldateien sowie Dateitypen von der Versionierung ausgeschlossen werden.
Weiter ermöglicht Git das Einbinden von Submodulen in ein Verzeichnis.
Dies kann über die Datei \verb*|.gitmodules| konfiguriert werden.

\subsection{Toolchain}\label{fig:toolchain}
Vom Programmcode zur Ausführung auf dem Mikrocontroller.
Die Programmiersprache C ist für den Menschen zwar gut lesbar, ein Mikrocontroller kann damit allerdings nichts anfangen.
Dieser benötigt eine Datei mit Anweisungen in einer für ihn verständlichen Sprache.
Das bedeutet, es wird ein Bitmuster mit den entsprechenden Befehlen benötigt \cite{Flik2004-dl}.
Dieses sogenannte Maschinenprogramm kann in den Speicher des Mikrocontrollers geladen und ausgeführt werden. \\

Das Programmieren mit Assembler oder gar in Maschinensprache ist eine Seltenheit geworden.
Kleine Änderungen müssen an diversen Stellen berücksichtigt werden, damit das Programm weiterhin lauffähig bleibt \cite{Flik2004-dl}.
Auch die Portabilität des Programmcodes ist ein Problem.
Assembler Befehlssätze sind auf eine Rechnerarchitektur angepasst.
Änderungen der Hardware führen daher zwangsweise zu einer Anpassung des Programms.
Stattdessen wird in einer höheren Programmiersprache der Programmcode entwickelt.
Bei einer Hochsprache kann durch den Compiler der Maschinencode für verschiedene Prozessoren erzeugt werden \cite{Wust2010-jg}.\\

Das Umwandeln des Programmcodes aus der Hochsprache benötigt für ein lauffähiges Image mehr als nur den Compiler.
Vor dem Compiler werden Definitionen im Programmcode durch ihre Werte ersetzt.
Diese Aufgabe übernimmt der Präprozessor.
Nach dem Compileraufruf sorgt der Linker dafür, dass aus allen generierten Objektdateien eine ausführbare Datei erstellt wird.
Dieser Ablauf der verschiedenen Werkzeuge wird Toolchain genannt. \\

Damit der Ablauf der Toolchain gewährleistet ist, muss diese konfiguriert werden.
Dafür gibt es spezielle Dateien.
Die Linker-Konfigurationsdatei ist für den weiteren Verlauf dieser Thesis von besonderer Bedeutung.
Der Ablauf der Toolchain für die Programmiersprache C ist in Abbildung \ref{fig:ablaufcompiler} dargestellt \cite{Küveler2013}.\\

\begin{figure}[H]
	\centering
	\includegraphics[width=1\linewidth]{graphics/ablauf_compiler}
	\caption{Ablauf der Toolchain}
	\label{fig:ablaufcompiler}
\end{figure}

Dieser Ablauf wird beim Starten des Übersetzungsvorganges durchlaufen.
\begin{description}[style=nextline]
	\item[Präprozesssor] Ersetzt die im Programmcode beschriebenen Definitionen durch die entsprechenden Werte.
	Er durchsucht den Quelltext nach für ihn bestimmten Anweisungen.
	Der Quellcode wird durch den Präprozessor modifiziert, bevor es mit dem Compile-Vorgang weiter geht \cite{Klima2010-br}.
	Gekennzeichnet werden die Anweisungen für den Präprozessor mit dem Symbol \#.
	
	\item[Compiler] Der Compiler übersetzt ein Programm von einer Quellsprache in eine Zielsprache \cite{Watson2017}.
	In diesem Fall wird von C in Maschinencode übersetzt.
	Ein Zwischenschritt zu Assembler ist dabei auch möglich.
	Dies hängt vom Compiler bzw. dessen Konfiguration ab.\\
	Während des Compiler-Vorgangs kann der Programmcode außerdem optimiert werden.
	Dabei gibt es verschiedene Optimierungsverfahren.
	Es kann zum Beispiel eine Optimierung nach Laufzeit stattfinden.
	Dabei werden gegebenenfalls die Assemblerbefehle umsortiert, ohne die Logik des Programms zu ändern \cite{Klima2010-br}.
	Bei der \acs{SR}-Software der \acsp{ECU} wird der Schritt des Optimierens allerdings übersprungen.
	Ein Software-Projekt besteht im Allgemeinen aus mehreren Quelltextdateien.
	Für jede dieser Dateien wird der Präprozessor- und Compiler-Prozess durchlaufen.
	Dadurch wird für jede Datei eine Objekt-Datei erzeugt \cite{Klima2010-br}.
	
	\item[Linker] Der Linker fügt die Objekt-Dateien zu einem ausführbaren Programm zusammen.
	Weitere Informationen können in Kapitel \textit{\ref{kap:aufgabe_linker} \nameref{kap:aufgabe_linker}} zu diesem Schritt gefunden werden.
\end{description}
Der letzte Schritt in der Welt der Embedded-Geräte ist das Programmieren des Programmcodes auf den Mikrocontroller.
Zum Programmieren wird eine entsprechende Schnittstelle benötigt.
Dann kann das ausführbare Programm auf den Controller geladen und gestartet werden.
Passende Programmiergeräte können durch den Hersteller der Controller zur Verfügung gestellt werden \cite{Wust2010-jg}.
Die Software wird dabei in den Speicher des Mikrocontrollers geladen.
Typischerweise im Flash-Speicher.
Dadurch ist ein schnelles und einfaches Aufspielen möglich \cite{Wust2010-jg}.
Dieser Schritt ist in der Abbildung \ref{fig:ablaufcompiler} nicht dargestellt. \\

Der Linker sorgt dafür, dass alle Teile des Programms korrekt zusammengefügt werden.
Ein tieferes Verständnis ist für den weiteren Verlauf der Thesis wichtig.
Daher wird im nächsten Kapitel \textit{\ref{kap:linker} \nameref{kap:linker}} tiefer auf die Funktion und Konfiguration eingegangen.