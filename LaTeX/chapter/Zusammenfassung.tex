\section{Zusammenfassung}\label{kap:zusammenfassung}

Die graphische Anwendung \verb*|Linker| \verb*|File| \verb*|Analyser| wird durch die Nutzung verschiedener Werkzeuge und Best Practices erweitert. 
Zur Verbesserung der Übersichtlichkeit der \ac{GUI} werden die Fenster gruppiert und in einem Notizbuch mit Registerkarten dargestellt. 
Mit der Software können Konfigurationen für die \ac{MPU} erstellt, eingelesen und visuell präsentiert werden. 
Das Design dieser Registerkarten bleibt dabei weitgehend unverändert. \\

Die Software ermöglicht das Einlesen und Analysieren von Linker-Konfigurationsdateien. 
Der Algorithmus wird angepasst, um Segmentbezeichnungen zu erkennen und diesen Speicherbereichen in einer Datenstruktur zuzuweisen. 
Informationen wie Start- und Endadressen werden in hexadezimaler Form angezeigt. 
Zur Erweiterung der Funktionalität wird die Oberfläche der Registerkarte für das Einlesen der Linker-Konfigurationsdateien umgestaltet.
Nutzer können über die \ac{GUI} Speicherbereiche anlegen, verändern oder löschen, ohne direkt in der Datei suchen zu müssen.
Segmente können ebenfalls angelegt, aber nicht verändert oder gelöscht werden.
Änderungen werden durch einen Algorithmus in der Datei übernommen, der diese erkennt und aktualisiert. \\

Obwohl die Artefaktzuweisung der Speicherregionen eine wichtige normative Anforderung darstellt, konnte sie aufgrund des begrenzten Zeitrahmens nicht umgesetzt werden. 
Diese Funktion bietet jedoch Potenzial für eine zukünftige Implementierung in verbesserter Form. \\

Die durchgeführten Erweiterungen haben die Benutzerfreundlichkeit und Funktionalität der Software erheblich verbessert. 
Zukünftige Verbesserungen, Bugfixes und funktionale Erweiterungen sind möglich. 
Die vollständige Erfüllung der Anforderungen dieser Thesis, insbesondere die Artefaktzuweisung der Speicherregionen, stellt eine wichtige Erweiterung für die Zukunft dar. 
Die Darstellung eines Änderungsprotokolls in der \ac{GUI} könnte die Transparenz für Nutzer erhöhen, da aktuell Änderungen erst nach dem Aktualisieren der Datei sichtbar sind. 
Eine eigene Registerkarte für diese Funktion ist denkbar. \\

Zukünftig können die Speicherregionen in der \ac{GUI} sortiert angezeigt werden, um eine bessere Übersicht zu gewährleisten.
Momentan erfolgt die Anordnung in der Reihenfolge der Definitionen in der Datei. 
Kontinuierliche Verbesserungen und Erweiterungen können die Benutzerfreundlichkeit langfristig steigern und den Funktionsumfang der Software erweitern.