\section{Verwendung Linker File Analyser}\label{kap:verwendung_lfa}
\subsection{Öffnen von Linker-Konfigurationsdateien}
Nach dem Starten der Software wird der Nutzer mit einer graphischen Benutzeroberfläche begrüßt.
Diese kann in Abbildung \ref{fig:hauptanwendungsfensterleer} gesehen werden.
Darin ist die Funktionalität der Software teilweise limitiert.
Diese Limitierungen werden aufgelöst, sobald eine Linker-Konfigurationsdatei eingelesen wird. 
Die Registerkarte \verb*|Visualize| \verb*|Config| ist dabei uneingeschränkt nutzbar. \\
\begin{figure}[H]
	\centering
	\includegraphics[width=0.95\linewidth]{graphics/hauptanwendungsfenster_leer}
	\caption{Hauptanwendungsfenster ohne Regionen}
	\label{fig:hauptanwendungsfensterleer}
\end{figure}

Über die Schaltfläche \verb*|Open| kann ein Dialogfenster zur Auswahl der Datei geöffnet werden.
Darin kann eine passende Linker-Konfigurationsdatei ausgewählt werden.
Das Dialogfenster ist in Abbildung \ref{fig:filedialog} dargestellt.
Limitiert ist die Auswahl des Dateityps.
Lediglich \verb*|icf|-Dateien können ausgewählt werden. \\

\begin{figure}[H]
	\centering
	\includegraphics[width=0.75\linewidth]{graphics/file_dialog}
	\caption{Dialogfenster Datei auswählen}
	\label{fig:filedialog}
\end{figure}

Nach dem Öffnen einer Datei werden alle inaktiven Schaltflächen auf aktiv gestellt.
Damit steht dem Nutzer die volle Funktionalität der Software zur Verfügung.
Die eingelesenen Regionen werden wie in Abbildung \ref{fig:hauptanwendungsfensterupdated} auf Seite \pageref{fig:hauptanwendungsfensterupdated} dargestellt.
Die weiteren Funktionsabläufe werden in den beiden folgenden Kapiteln näher beschrieben.

\subsection{Editieren von Linker-Konfigurationsdateien}
Wie im vorherigen Kapitel beschrieben, können Linker-Konfigurationsdateien geöffnet werden.
Die darin enthaltenen Speicherregionen werden über die graphische Oberfläche dargestellt.
Das generelle Aussehen der GUI kann auf den Abbildungen \ref{fig:hauptanwendungsfensterleer} (Seite \pageref{fig:hauptanwendungsfensterleer}) und \ref{fig:hauptanwendungsfensterupdated} (Seite \pageref{fig:hauptanwendungsfensterupdated}) gesehen werden.\\

Die Speicherregionen können vom Benutzer angepasst werden.
Es kann die Startadresse sowie die Größe einer Region geändert werden.
Dies kann über die Textfelder geschehen.
Die Farbe Orange zeigt dabei die Textfelder an, welche editierbar sind.
Über die Schaltfläche \verb*|Update| können Änderungen in die Linker-Konfigurationsdatei übernommen werden. \\

Weiter können in diesem Fenster neue Speicherregionen angelegt werden.
Dafür kann auf der rechten Seite der \ac{GUI} der Name sowie Startadresse und Speichergröße definiert werden.
Darunter kann ausgewählt werden, in welchem Segment die Region angelegt werden soll.
Auch hierbei wird die Änderung erst beim Aktualisieren in die Datei geschrieben. \\

Als letzte Funktion kann ein Segment angelegt werden.
Die dazugehörigen Optionen können unterhalb des Bereichs für das Anlegen von Speicherregionen gefunden werden.
Dabei kann der Name definiert werden.
Außerdem kann ausgewählt werden, ob der Bereich über die \ac{IDE} auswählbar sein soll.

\subsection{Erstellen und Visualisieren von MPU-Konfigurationen}
Mithilfe der \ac{MPU} ist es möglich, Speicherbereiche vor ungewolltem Beschreiben zu schützen.
Definiert in der Linker-Konfigurationsdatei sind diese Bereiche über die \ac{GUI} dargestellt.
In der Registerkarte \verb*|Linker| \verb*|File| \verb*|Editor| können die unterschiedlichen Speicherbereiche vorgemerkt werden.
Die Registerkarte \verb*|Export| \verb*|Config| bietet die Funktionen, um Konfigurationen für die \ac{MPU} zu erstellen. \\

Dabei werden auf der rechten Seite zunächst die vorgemerkten Speicherbereiche aufgelistet.
Diese Auflistung wird bei jedem Aufruf der Registerkarte aktualisiert.
Auf der linken Seite der Registerkarte befinden sich Optionen.
Als Erstes kann dort der Speicherort eingestellt werden.
Darunter kann entweder ein Dateiname aus den ausgewählten Regionen erstellt oder ein eigener Name angelegt werden.
Die Mikrocontroller-Architektur kann ebenfalls ausgewählt werden.
Dafür stehen die vom Skript unterstützen über ein Auswahl-Widget zur Verfügung.
Mit dem Schalter in dem \verb*|Combine|-Labelframe kann zwischen zwei Modi gewält werden.
Der erste Modus erstellt aus allen Regionen eine einzelne Konfiguration.
Dies funktioniert allerdings nur bei zusammenhängenden Speicherbereichen.
Als zweiter Modus wird für jede Speicherregion eine eigene Konfiguration erstellt. \\

Über die Schaltfläche \verb*|Create| wird der Prozess des Erstellens der Konfiguration gestartet.
Sollte dabei ein Fehler auftreten, wird dies dem Nutzer über ein Nachrichtenfenster angezeigt.
Bei erfolgreichem Durchlauf des Skripts wird eine Erfolgsnachricht ausgegeben.
Die Konfiguration wird in einer Header-Datei gespeichert.
Darin werden die Einträge eines Arrays generiert.
Das Array wird weiterhin von Nutzer manuell angelegt.
Damit kann die Bezeichnung des Arrays frei gewählt werden.
Die Header-Datei kann an der richtigen Stelle in den Source-Code eingebunden und in die Register der \ac{MPU} geladen werden.
In Programmcodeabschnitt \ref{pc:beispielkonfig} kann eine solche Header-Datei gefunden werden. \\

Darin werden unterschiedliche Kommentare eingepflegt.
Diese dienen dazu, bei der Visualisierung die Regionen und Subregionen korrekt darzustellen.
In der generierten Header-Datei befindet sich zusätzlich eine Beschreibung und Warnung, diese Datei nicht manuell zu bearbeiten.
Im Beispiel aus Programmcodeabschnitt \ref{pc:beispielkonfig} ist dies weggelassen, weil es nicht weiter für das Verständnis der Dateistruktur beiträgt.

\begin{listing}[H]
	\begin{minted}[autogobble, linenos, numberblanklines=true, frame=leftline, framesep=5pt, numbersep=5pt, style=vs, tabsize=3, xleftmargin=10pt, fontsize=\footnotesize, baselinestretch=1.25, breaklines]{c}
		#ifndef MPU_REGIONS_H
		#define MPU_REGIONS_H
		// ------------------------------------------------------------------------
		// FILE PROPERTIES (PROTECTED):
		// PROTECTED MEMORY AREAS: SR_DATA_CRC|SR_DATA|SR_ISR_DATA|SR_ISR_DATA_INV_RED
		// ARCHITECTURE: M0+ | START: 0x20002600 | END: 0x20003DFF |  SIZE: 0x1800
		// ------------------------------------------------------------------------
		
		// REGION_NUMBER: 1
		// PROTECTED REGION: START: 0x20002600 | END: 0x20002FFF | SIZE: 0xA00 (5 * 0x200)
		// ACTUAL: START: 0x20002000 | SIZE: 0x1000 | SRD: 0x07
		{ .region_address        = (0x20002000UL),
			.region_flags_and_size = ((MPU_REGION_SIZE_4KB << MPU_RASR_SIZE_Pos)
			| (MPU_RASR_ENABLE_Msk << MPU_RASR_ENABLE_Pos)
			| MPU_RASR_S_Msk
			| (MPU_REGION_PRIV_RO_URO << MPU_RASR_AP_Pos)
			| (0x07 << MPU_RASR_SRD_Pos))
		},
		
		// REGION_NUMBER: 2
		// PROTECTED REGION: START: 0x20003000 | END: 0x20003DFF | SIZE: 0xE00 (7 * 0x200)
		// ACTUAL: START: 0x20003000 | SIZE: 0x1000 | SRD: 0x80
		{ .region_address        = (0x20003000UL),
			.region_flags_and_size = ((MPU_REGION_SIZE_4KB << MPU_RASR_SIZE_Pos)
			| (MPU_RASR_ENABLE_Msk << MPU_RASR_ENABLE_Pos)
			| MPU_RASR_S_Msk
			| (MPU_REGION_PRIV_RO_URO << MPU_RASR_AP_Pos)
			| (0x80 << MPU_RASR_SRD_Pos))
		},
		#endif
	\end{minted}
	\caption{MPU-Konfigurationsdatei}
	\label{pc:beispielkonfig}
\end{listing}

Für die Visualisierung einer solchen Konfiguration steht die Registerkarte \verb*|Visualize| \verb*|Config| zur Verfügung.
Die Schaltflächen unter dem Labelframe-Widget mit dem Text \verb*|Regions| bietet verschiedene Optionen.
Über die Schaltfläche \verb*|Add| kann eine Konfigurationsdatei ausgewählt werden.
Dafür öffnet sich ein ähnliches Fenster wie aus Abbildung \ref{fig:filedialog} (Seite \pageref{fig:filedialog}).
Weiter bietet die Schaltfläche \verb*|Last| \verb*|Config| die Möglichkeit die zuletzt erstellte Konfiguration zu laden.
Ist noch keine erstellt worden, wird dem Nutzer eine entsprechende Meldung angezeigt.
Zuletzt kann über die Schaltfläche \verb*|Clear| alle angezeigten Darstellungen der Konfigurationen aus der Registerkarte gelöscht werden. \\

Die Abbildung \ref{fig:visualizewindow} (Seite \pageref{fig:visualizewindow}) zeigt dabei die Darstellung der Konfigurationsdatei aus Programmcodeabschnitt \ref{pc:beispielkonfig}.
Die Kommentare der Datei geben die Informationen über die Bezeichnung der Regionen, die Startadresse, Größe sowie Endadresse.
Dies wird in dem Informationsfeld links neben einer Darstellung der Konfiguration angezeigt.
Rechts daneben ist aufgeschlüsselt, welche Subregionen aktiv bzw. nicht aktiv sind.
Eine entsprechende Farblegende ist unten rechts zu finden.